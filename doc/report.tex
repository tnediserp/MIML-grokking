\documentclass{article}


% if you need to pass options to natbib, use, e.g.:
%     \PassOptionsToPackage{numbers, compress}{natbib}
% before loading neurips_2022


% ready for submission
\usepackage[preprint,nonatbib]{neurips_2022}


% to compile a preprint version, e.g., for submission to arXiv, add add the
% [preprint] option:
%     \usepackage[preprint]{neurips_2022}


% to compile a camera-ready version, add the [final] option, e.g.:
%     \usepackage[final]{neurips_2022}


% to avoid loading the natbib package, add option nonatbib:
% \usepackage[nonatbib]{neurips_2022}


\usepackage[utf8]{inputenc} % allow utf-8 input
\usepackage[T1]{fontenc}    % use 8-bit T1 fonts
\usepackage{hyperref}       % hyperlinks
\usepackage{url}            % simple URL typesetting
\usepackage{booktabs}       % professional-quality tables
\usepackage{amsfonts}       % blackboard math symbols
\usepackage{nicefrac}       % compact symbols for 1/2, etc.
\usepackage{microtype}      % microtypography
\usepackage{xcolor}         % colors
\usepackage{graphicx}
\usepackage{subcaption}
\usepackage{amsthm,amsmath,amssymb}
\usepackage[noabbrev,capitalise]{cleveref}

\theoremstyle{definition}
\newtheorem{definition}{Definition}[section]

\newcommand{\ZZ}{\mathbb{Z}}


\title{Final Project: Grokking Phenomenon Reproduction}


% The \author macro works with any number of authors. There are two commands
% used to separate the names and addresses of multiple authors: \And and \AND.
%
% Using \And between authors leaves it to LaTeX to determine where to break the
% lines. Using \AND forces a line break at that point. So, if LaTeX puts 3 of 4
% authors names on the first line, and the last on the second line, try using
% \AND instead of \And before the third author name.


\author{%
  Kecen Sha \thanks{Equal contribution.} \\
  2200010611\\
  \And
  Yuziheng Wu \footnotemark[1]\\
  Your ID \\
  \And
  Di Yue \footnotemark[1]\\
  2100012961 \\
  % \And
  % Coauthor \\
  % Affiliation \\
  % Address \\
  % \texttt{email} \\
  % \And
  % Coauthor \\
  % Affiliation \\
  % Address \\
  % \texttt{email} \\
}


\begin{document}


\maketitle

\begin{abstract}
    % Grokking, a phenomenon that the generalization of a neural network happens much later than the convergence of its training loss, has received increasing attention from both learning theory and application since it was first introduced by [Power et al., 2022].
    We reproduce the grokking phenomenon [Power et al., 2022], that a neural network generalizes long after it memorizes the training data, for modular addition problem, and provide an explanation based on [Kumar et al., ICLR 2024].
    \footnote{All the codes and supplementary materials are available at: \url{https://github.com/tnediserp/MIML-grokking}}
\end{abstract}

\section{Introduction}
\label{sec:intro}

In the field of machine learning, it is an important goal to understand the generalization dynamics of neural networks.
Typically, the model's generalization ability can be well reflected by its performance on the training data.
However, when training neural networks for certain problems, generalization may occur long after the model overfits the training data. 
This striking phenomenon is called \emph{grokking}~\cite{Grokking}. 
Ever since it was first introduced, exhausted efforts have been made to understand grokking from a theoretical viewpoint.
For example, grokking is explained by the hardness of representation learning~\cite{LiuKNMTW22},
the effect of weight decay and weight norm decrease~\cite{Grokking_circuit_efficiency,LiuMT23}
and the transition between different training regimes~\cite{KumarBGP24,MohamadiLWS24}.

Among those learning tasks where grokking can be observed, the most well-studied one might be the \emph{modular addition problem}~\cite{Grokking,KumarBGP24,MohamadiLWS24,Gromov}.
Specifically, given a fixed prime number $p$, consider learning the output of the modular arithmetic: % where each equation consists of $K$ summands and is represented as a string:
\begin{equation}
    a_{1} + a_{2} + \cdots + a_{K} = b,
    \label{eqn:modular_addition}
\end{equation}
where both the individual summands $a_i$ and the sum $b$ are elements of the finite field $\mathbb{Z}/p\mathbb{Z}$.
We model \eqref{eqn:modular_addition} as a classification problem, where classes are labeled by integers $b \in \{0, 1, \dots, p-1\}$.
We mainly focus on the case $K=2$, and discuss general $K$-wise addition in \cref{sec:subtask4}.

% Our experiments are divided into 4 parts in section 3. \ref{sec:subtask1} reproduces the grokking phenomenon stated in \cite{Grokking} and study the effect of $\alpha$ on grokking phenomenon. \ref{sec:subtask2} uses other models such as LSTM and MLP to conduct the same task. \ref{sec:subtask3} investigates the impact of different optimizers(with different hyper-Parameters such as learning rate, weight decay and dropout) to grokking phenomenon. \ref{sec:subtask4} explores the grokking phenomenon for different $K$.  

% At the end of our report we give some explanations on grokking phenomenon, mainly based on \cite{JacotHG18}.


\section{Preliminary}
\label{sec:prelim}

\section{Experiments}
\subsection{Reproducing the Grokking Curve}
\label{sec:subtask1}

In this task, we investigate the grokking phenomenon in modular addition with the transformer model.
Specifically, we use a decoder-only transformer with $2$ layers, width $d_{\mathrm{model}} = 128$, $4$ attention heads and dropout $0.1$.
We use the AdamW optimizer with $\beta_1 = 0.9, \beta_2 = 0.98$, learning rate $10^{-3}$, weight decay $0.1$, and linear learning rate warmup over the first $10$ updates.
For $p = 97$ and $\alpha = 0.5$, we randomly separate an $\alpha$ fraction from all $p^2$ equations in $\ZZ_p$ as the training set; the rest serves as the validation set.
The model is trained with minibatch size $512$ for $10^5$ steps.

The accuracy and loss throughout the training process are plotted in \cref{fig:acc_and_loss_transformer}.

\begin{figure}[!ht]
    \centering
    \begin{subfigure}{0.45\textwidth}
        \centering
        \includegraphics[width=\linewidth]{fig/grokking_curves/addition_50_Transformer_step.png}
        \caption{Training and validation accuracy}
        \label{fig:grokking_curve_transformer}
    \end{subfigure}
    %\hfill
    \begin{subfigure}{0.45\textwidth}
        \centering
        \includegraphics[width=\linewidth]{fig/loss_curves/addition_50_Transformer_step.png}
        \caption{Training and validation loss}
        \label{fig:loss_curve_transformer}
    \end{subfigure}

    \caption{The grokking curves of transformer model}
    \label{fig:acc_and_loss_transformer}
\end{figure}

As is shown in \cref{fig:grokking_curve_transformer}, the model overfits the training data within $10^3$ updating steps.
Nevertheless, generalization does not happen until after $10^4$ steps.
\cref{fig:loss_curve_transformer} further illustrates that the decrease of loss is consistent with the increase of accuracy, both for training and validation phases.
Interestingly, while the training loss decreases monotonically, an increase of the validation loss is observed before it begins to converge.

We further study the effect of the training data fraction $\alpha$. 
For each $\alpha$ we sample, we train the model on a random $\alpha$ fraction of all $p^2$ equations for at most $10^5$ steps, and determine the minimum number of steps required to achieve validation accuracy $\geq 99\%$.
The results are plotted in \cref{fig:transformer_alpha}.
As a comparison, we plot the accuracy curves for $\alpha = 80\%$ and $\alpha = 33\%$ in \cref{fig:grokking_alpha_80,fig:grokking_alpha_33}, respectively.

\begin{figure}[!ht]
    \centering
    \begin{subfigure}[t]{0.3\textwidth}
        \includegraphics[width=\linewidth]{fig/Transformer_alpha/Transformer_alpha.png}
        \caption{Minimal number of training steps required to achieve validation accuracy $\geq 99\%$}
        \label{fig:transformer_alpha}
    \end{subfigure}
    \hfill %
    \begin{subfigure}[t]{0.3\textwidth}
        \includegraphics[width=\linewidth]{fig/grokking_curves/addition_80.0_Transformer_step.png}
        \caption{The accuracy curve for $\alpha = 80\%$}
        \label{fig:grokking_alpha_80}
    \end{subfigure}
    \hfill %
    \begin{subfigure}[t]{0.3\textwidth}
        \includegraphics[width=\linewidth]{fig/grokking_curves/addition_33.0_Transformer_step.png}
        \caption{The accuracy curve for $\alpha = 33\%$}
        \label{fig:grokking_alpha_33}
    \end{subfigure}

    \caption{Effect of the training data fraction $\alpha$}
    \label{fig:effect_of_alpha}
\end{figure}

When $\alpha = 80\%$, the model generalizes in $10^4$ steps.
As $\alpha$ decreases, it becomes easier for the model to overfit the training data, while the number of steps required for generalization increases rapidly.
When $\alpha \leq 30\%$, the model would not generalize in $10^5$ steps.
\subsection{Grokking Phenomenon of Other Models}
\label{sec:subtask2}
\subsection{Effects of Different Hyperparameters}
\label{sec:subtask3}

In this task,we firstly choose 9 different settings to study the effects of different hyperparameters on the transformer model,which are shown in \cref{fig:different_settings}.Then,we change 8 different weight decay values to study its impact on training the transformer model especially,which is shown in \cref{fig:different weight_decay}.With a parameter setting of p=97, we employ a step budget of 1e5, a decision inspired by the original paper's use.Similarly,the training accuracy raise up to 99\% within 1e3 in most experiments,but the validation accuracy differ from each other.

\begin{figure}[!ht]
    \centering
    \includegraphics[width=0.9\textwidth]{fig/different_settings/different_settings.png}
    \caption{Grokking phenomenon for different settings}
    \label{fig:different_settings}
\end{figure}
\begin{figure}[!ht]
    \centering
    \includegraphics[width=0.9\textwidth]{fig/weight_decay/weight_decay.png}
    \caption{Grokking phenomenon for different weight decay}
    \label{fig:different weight_decay}
\end{figure}

In \cref{fig:different_settings},the AdamW in baseline can do well when alpha is just 33\%,but the AdamW with 0.1x baseline learning rate performs poorly even when alpha is 60\%,which means small learning rate may cause slow training.However,big learning rate may not necessarily lead to fast training,which can be seen from the comparison between SGD in baseline(weight decay=0) and SGD with 10x baseline learning rate.Additionally,removing dropout has little impact on AdamW, but adding weight decay makes SGD's training even worse,which are all the characteristics of methods themselves.
As for the comparison between methods, under the same learning rate and dropout settings, AdamW is the best and SGD with Nesterov acceleration is the worst. RMSprop and mini batch Adam have acceptable performance when alpha is greater than 50\%, while full batch Adam needs to exceed 70\% to reach the same level.
In \cref{fig:different weight_decay},we select a column of gradually increasing weight decay values and train them on the transformer model using AdamW. As the weight decay value gradually increases, the number of training steps required to reach validation accuracy close to 100\% decreases, and the grokking phenomenon becomes less obvious. However, it is worth noting that when the weight decay value exceeds 0.2, there are occasional "rollback" phenomena in training accuracy and validation accuracy on certain training steps, which occur more frequently as the weight decay value increases. When the weight decay value is 1.5 and the training steps exceed 3e4, accuracy cannot even be restored. The reason is worth further exploration, but at least it tells us that the selection of training steps does not need to be too large.
\subsection{Grokking for $K$-Wise Modular Addition}
\label{sec:subtask4}

\section{An Explanation of the Grokking Phenomenon}
\label{sec:explanation}

In this section, we provide an explanation of the grokking phenomenon based on~\cite{KumarBGP24}, which claims that grokking happens as a transition between different regimes of training.
We first briefly review the kernel regime and rich regime defined in~\cite{KumarBGP24} in \cref{subsec:regimes}, and illustrate how they help explain the grokking phenomenon for modular addition in \cref{subsec:explain_by_regimes}.

\subsection{Kernel Regime And Rich Regime}
\label{subsec:regimes}

We have the following definition of neural tangent kernel (NTK).

\begin{definition}[Neural Tangent Kernel~\cite{JacotHG18,KumarBGP24}]
    \label{def:NTK}
    Let $\Theta$ be the parameter space and $\mathcal{X}$ be the input space.
    Let $f \colon \Theta \times \mathcal{X} \to \mathcal{Y}$ be a neural network.
    For $\theta \in \Theta$, the \emph{neural tangent kernel} of $f(\theta, \cdot)$ is defined as 
    \begin{align*}
        K_\theta(\mathbf{x}, \mathbf{x}') := \nabla_\theta f(\theta, \mathbf{x}) \nabla_\theta f(\theta, \mathbf{x}')^\top, 
        \quad \forall \mathbf{x}, \mathbf{x}' \in \mathcal{X}.
    \end{align*}
\end{definition}

In the \emph{kernel regime}, we have the following approximation of the model output, 
\begin{align*}
    f(\theta, \mathbf{x}) \approx f(\theta_0, \mathbf{x}) + \Braket{\nabla_{\theta} f(\theta_0, \mathbf{x}), \theta - \theta_0}.
\end{align*}
Note that the right-hand side is linear in $\theta$, which means that $f(\theta, \cdot)$ has approximately the same expressivity as the kernel method with respect to $K_{\theta_0}$, as defined in \cref{def:NTK}.
Under gradient descent, the model's parameters $\theta$ are restricted to the affine subspace $W := \mathrm{span}\{\nabla_{\theta} f(\theta_0, \mathbf{x}_i)\}_{i=1}^n$.
Therefore, the model will first converge to the local optimal solution $\theta_W^* \in W$ in the kernel regime.
\emph{This corresponds to the stage where the model overfits the training data but does not generalize.}

When the model begins to learn important non-linear features, it will escape $W$ and enter the \emph{rich regime}, where it finally converges towards the global optimal solution $\theta^*$.
\emph{This corresponds to the generalization phase in the grokking phenomenon.}

It is worth mentioning that normalization methods such as weight decay accelerates generalization by encouraging the model to escape from the kernel regime, which is consistent with our experiment results in \cref{sec:subtask3}.


\subsection{Regime Transition in Modular Addition Problem}
\label{subsec:explain_by_regimes}

We further illustrate how grokking is related to the transition between the aforementioned two regimes in the modular addition problem.
We first rewrite the target function $f^*$ as 
\begin{align*}
    f^*(e_a, e_b) = e_{a+b} = H^a e_b = H^b e_a, 
\end{align*}
where $H = \sum_{j=0}^{p-1} e_{j+1} e_{j}^\top$ is the $p \times p$ cyclic permutation matrix.
When the first (second) coordinate of $f^*$ is fixed, it becomes a simple linear function of the second (first) coordinate.
Following this observation, we consider the following generalized version of $f^*$.

Let $A$ be any non-empty subset of $\ZZ_p$, and $E(A) := \{e_a \colon a \in A\}$. Denote $E(\ZZ_p) := \{e_0, e_1, \dots, e_{p-1}\}$.
Define $f_A^* \colon E(A) \times E(\ZZ_p) \to E(\ZZ_p)$ to be the restriction of $f^*$ on $E(A) \times E(\ZZ_p)$.
Intuitively, the smaller $\norm{A}$ is, the closer $f_A^*$ is to a linear map.
We define $\lambda := \frac{\norm{A}}{p} \in (0, 1]$ to be a measure of the ``non-linearity'' of $f_A^*$.
Specifically, $f^* = f_{\ZZ_p}^*$, and thus has non-linearity $1$.

To prove the explanations in \cref{subsec:regimes}, we study how the non-linearity $\lambda$ influences the grokking phenomenon.
For each given $\lambda$, we sample a random subset $A \subseteq \ZZ_p$ of size $\lambda p$.
We then train a $2$-layer MLP on $A \times \ZZ_p$ with training data fraction $\alpha = 0.6$, SGD optimizer and weight decay $0$ to learn the target function $f_A^*$.
The results are shown in \cref{fig:acc_and_loss_different_lambda}.

\begin{figure}[!ht]
    \centering
    \begin{subfigure}{0.45\textwidth}
        \centering
        \includegraphics[width=\linewidth]{fig/grokking_curves/different_Afraction_acc.png}
        \caption{Training and validation accuracy}
        \label{fig:different_lambda_acc}
    \end{subfigure}
    %\hfill
    \begin{subfigure}{0.45\textwidth}
        \centering
        \includegraphics[width=\linewidth]{fig/loss_curves/different_Afraction_loss.png}
        \caption{Training and validation loss}
        \label{fig:different_lambda_loss}
    \end{subfigure}

    \caption{Learning curves of different target functions $f_A^*$.}
    \label{fig:acc_and_loss_different_lambda}
\end{figure}

\cref{fig:different_lambda_acc} shows that it becomes slower to overfit the training data as non-linearity $\lambda$ increases.
Perhaps surprisingly, generalization nevertheless becomes easier as $\lambda$ increases.
For $\lambda = 0.1$, the model does not generalize at all and the validation loss blows up.
For $\lambda = 0.3$ and $0.5$, some generalization happens, but the validation loss does not fully converge within $10^5$ steps.
For $\lambda = 1.0$, the model quickly generalizes after it begins to fit the training data, and the grokking phenomenon almost vanishes.

We believe this interesting result reflects the mechanism of regime transition.
For smaller $\lambda$'s, the target function $f_A^*$ behaves more like a linear function.
Hence, the model tends to first learn these linear features.
It is then trapped in the kernel regime, and have to take a long time to escape from the affine subspace $W$ and enter the rich regime;
this is how grokking happens.
To be specific, the peaks of the loss curves in \cref{fig:different_lambda_loss} approximately corresponds to the kernel regime, and the similar shapes can also be observed in \cref{fig:loss_curve_transformer,fig:loss_curve_LSTM,fig:loss_curve_MLP}.
In contrast, For a larger $\lambda$, the function $f_A^*$ is very far from a linear map, hence forcing the model to learn other useful features.
Therefore, the model quickly moves to the rich regime and converges to the global optimal solution.
Since the transition happens immediately, grokking almost disappears in this setting.


\addcontentsline{toc}{section}{References}
\bibliographystyle{unsrt}
\begin{small}	
    \bibliography{ref}
\end{small}


%%%%%%%%%%%%%%%%%%%%%%%%%%%%%%%%%%%%%%%%%%%%%%%%%%%%%%%%%%%%


\appendix


\section{Appendix}


Optionally include extra information (complete proofs, additional experiments and plots) in the appendix.
This section will often be part of the supplemental material.


\end{document}
